\documentclass{spisok-article}

\title{Об устойчивости всплесковых разложений}

\author{Демьянович Ю.К., профессор кафедры параллельных алгоритмов СПбГУ, y.demjanovich@spbu.ru\\
  Иванцова О. Н., доцент кафедры параллельных алгоритмов СПбГУ, o.ivancova@spbu.ru
}

\begin{document}

\maketitle

\begin{abstract}
В работе устанавливаются критерии устойчивости сплайн-всплесковых разложений. Полученные критерии применимы к всплесковым разложениям нулевого, первого и второго порядков.
\end{abstract}

%\usepackage[T2A]{fontenc}
%\usepackage[cp1251]{inputenc}
%\usepackage[english,russian]{babel}
%\usepackage{multicol}
%\usepackage{graphicx}
%\usepackage{amsmath,amsthm, amsfonts}
%\usepackage{amssymb}
%\usepackage{listings}
%\usepackage{epsf,pstricks}
%\usepackage{graphicx}
%\usepackage{url}
%\usepackage{rotating}

\def\defabove{\raise3pt\hbox {${{\lower2pt  \hbox {${\scriptscriptstyle {\rm def}}$}}}  \atop = $}}

\def\aa{\hbox{\rm \bf a}}
\def\bb{\hbox{\rm \bf b}}
\def\cc{\hbox{\rm \bf c}}



\section{Введение}

Всплесковые (вэйвлетные) разложения широко используются при обработке числовых информационных потоков; объемы таких потоков постоянно возрастают, и это является стимулом к дальнейшему развитию теории всплесков (\cite{dem2003}, \cite{dem2013}). Используемый в данной работе подход к построению всплесков основывается на применении аппроксимационных соотношений, так что автоматически обеспечивается эффективная аппроксимация (чаще всего, она асимптотически оптимальна по $N$-поперечнику стандартных компактов). В противоположность классическим вэйвлетам  упомянутый подход позволяет без дополнительных сложных исследований использовать неравномерную сетку (как конечную, так и бесконечную), что весьма важно для экономии компьютерных ресурсов в случае появления сингулярных изменений рассматриваемых потоков. Кроме того, при построении всплесковых разложений в многомерном случае (и даже на произвольном дифференцируемом многообразии) могут применяться известные конечно-элементные аппроксимации (Куранта, Зламала и т.п.), что существенно расширяет возможности использования всплесковых разложений. В классическом случае большую трудность представляет построение всплескового (вэйвлетного) базиса в том или ином функциональном пространстве (часто в $L_2$(${  \mathbb{\mathbb{R}}}^1$)); используемый здесь подход не требует предварительного построения всплескового базиса (при желании этот базис может быть получен после проведения основных исследований). С другой стороны, знание всплескового базиса позволяет достичь существенной экономии компьютерных и сетевых ресурсов. Заметим, что для получения упомянутой экономии не нужен всплесковый базис в пространстве функций с континуальной областью определения, достаточно лишь получить подходящий базис для пространства всплесковых числовых потоков, но для этого необходимо все построения выполнять без использования функций с континуальной областью определения. 

Цель данной работы состоит в том, чтобы в рамках упомянутого подхода рассмотреть условия устойчивости сплайн-всплескового разложение относительно возмущения исходных данных. Для формул декомпозиции исходными данными является, так называемый, исходный поток. При рассмотрении формул реконструкции исходными данными являются основной и всплесковый потоки. 



\section{Предварительные сведения}


Рассмотрим линейное пространство $\mathbb{L}$ функций, определенных на интервале $(\alpha,\beta) \subset {\mathbb{\mathbb{R}}}^1$. На интервале $(\alpha,\beta)$ введем сетку
$$   
X: \ldots<x_{-2}<x_{-1}<x_{0}<x_{1}<x_{2}\ldots,
\eqno (1.1)
$$

$$
\lim_{j\to -\infty}x_j=\alpha,\quad
\lim_{j\to +\infty}x_j=\beta.
\eqno (1.2)
$$
  
Пусть $\varphi(t) \defabove (\varphi_0(t), \varphi_1(t), \ldots , \varphi_m(t))$ --- вектор-функция с компонентами, принадлежащими пространству $  \mathbb{L}$: $\varphi_i \in   \mathbb{L}$, $i \in \{0,1,\ldots, m\}$. 

Рассмотрим множество $G$ линейных функционалов $g^{(s)} \in { \mathbb{L}}^*$, $G \defabove \{g^{(s)}\}_{s \in   \mathbb{Z}}$, со свойством

$$ 
{\rm supp}\; g^{(s)} \subset (x_s, x_{s+1}) \quad\forall s \in   \mathbb{Z}.
\eqno (1.3)
$$ 
Результат действия функционала $g^{(s)}$ на функцию $u \in  \mathbb{L}$ обозначим острыми скобками $\langle g^{(s)}, u \rangle $. Упомянутые скобки будем использовать также для вектор-столбца с числовыми компонентами 

$$
\langle g^{(s)}, \varphi \rangle \defabove (\langle g^{(s)}, \varphi_0 \rangle, \langle g^{(s)}, \varphi_1 \rangle, \ldots ,\langle g^{(s)}, \varphi_m \rangle)^T.
$$ 

\section{Условие невырожденности. Полная цепочка векторов}

Предположим, что выполнено условие
$$
\det(\langle g^{(s)}, \varphi \rangle, \langle g^{(s+1)}, \varphi \rangle, \ldots ,\langle g^{(s+m)}, \varphi \rangle) \ne 0 \quad\forall s \in   \mathbb{Z}.
\eqno (2.1)
$$  
Условие (2.1) называется {\it условием невырожденности}.

Рассмотрим цепочку $m + 1$-мерных векторов $\ldots, \aa_{-2}, \aa_{-1}$, $\aa_0$, $\aa_1$, $\aa_2, \ldots$. Если при каждом фиксированном $s \in   \mathbb{Z}$ система векторов $\{\aa_s, \aa_{s+1}, \ldots, \aa_{s+m+1}\}$  линейно независимая, то цепочка векторов $\ldots, \aa_{-2}, \aa_{-1}, \aa_0, \aa_1, \aa_2, \ldots$ называется {\it полной}.

По определению положим 
$$
\aa_s \defabove \langle g^{(s)}, \varphi \rangle. 
\eqno (2.2)
$$

Цепочка векторов (2.2) --- полная, поскольку из предположения (1.3), (2.1) следует линейная независимость системы векторов $\aa_s, \aa_{s+1}, \ldots, \aa_{s+m+1}$  при любом фиксированном целом $s$. 

{\it Замечание}. Если $  \mathbb{L}$ является пространством непрерывных функций на интервале $(\alpha,\beta)$, и $\langle g^{(s)}, u \rangle \defabove u(x_s)$, $s \in   \mathbb{Z}$, а система функций $\{\varphi_0, \varphi_1, \ldots ,\varphi_m \}$ является системой Чебышева на $(\alpha,\beta)$, то свойство (2.1) выполнено. Заметим, что примером системы Чебышева на любом интервале $(\alpha,\beta) \subset  {\mathbb{R}^1}$ является система $\{1, t, \ldots , t_m\}$.

\section{Аппроксимационные соотношения и пространство минимальных сплайнов}

Теперь рассмотрим соотношения
$$
\sum_{i=k-m}^k \aa_i\omega_i(t) = \varphi(t) \quad\forall t \in (x_k, x_{k+1}) \quad \forall k \in   \mathbb{Z}, 
\eqno (3.1)
$$ 
$$
{\rm supp}\; \omega_s \subset [x_s, x_{s+m+1}] \quad\forall s \in   \mathbb{Z}.
\eqno (3.2)
$$
Соотношения (3.1) -- (3.2) называются {\it аппроксимационными соотношениями}. Ввиду предположения (2.1) из аппроксимационных соотношений однозначно определяются функции $\omega_i(t)$. При предположениях (2.1) эти функции образуют линейно независимую систему. Они называются координатными сплайнами. 

Предположим, что $\omega_s \in   \mathbb{L}$. Из формул (1.3), (2.1) -- (2.2), (3.1) -- (3.2) следуют соотношения биортогональности системы функционалов $\{g^{(j)}\}$ по отношению к системе координатных сплайнов $\{\omega_s\}_{s \in   \mathbb{Z}}$:
$$
\langle g^{(j)}, \omega_s \rangle = \delta_{j,s} \quad\forall j, s \in   \mathbb{Z}. 
$$
Рассмотрим линейное пространство $\mathbb S$, состоящее из линейных комбинаций координатных сплайнов, $\mathbb {S} = \mathbb {S}(X, \varphi) \defabove\mathcal{L} \{\omega_s \}_{s \in \mathbb{Z}}$, где через $\mathcal{L}$ обозначена линейная оболочка функций, находящихся в фигурных скобках. Пространство $(X, \varphi)$ называется {\it пространством минимальных сплайнов}. 

\section{Вложенное пространство минимальных сплайнов}

Пусть $\widetilde X$ --- подмножество сетки $X$ такое, что
$$
X: \ldots < \widetilde x_{-2} < \widetilde x_{-1} <\widetilde x_{0} < \widetilde x_{1} < \widetilde x_{2} < \ldots,
$$

$$
\lim_{j\to -\infty}\widetilde x_j=\alpha,\quad
\lim_{j\to +\infty}\widetilde x_j=\beta, \quad
\widetilde X \subset X.
$$

Предположим, что $\{\widetilde g^{(i)}\}_{i\in   \mathbb{Z}}$ --- система линейных функционалов со свойствами
$$
{\rm supp}\; \widetilde g^{(i)} \subset (\widetilde x_i, \widetilde x_{i+1}) \quad \det(\langle \widetilde g^{(i)}, \varphi \rangle) \ne 0 \quad \forall i \in   \mathbb{Z}. 
\eqno (4.1)
$$

Рассмотрим функции $\widetilde\omega_i$, связанные с новой сеткой $\widetilde X$ аналогично предыдущему, и введем линейное пространство минимальных сплайнов $\widetilde{\mathbb{S}}$, ассоциированное с новой сеткой $\widetilde{\mathbb{S}} = \mathbb{S}(\widetilde X,\varphi)$. 

Предположим, что пространство $\widetilde{\mathbb{S}}$ является подпространством пространства $\mathbb{S}$, так что
$$
\widetilde{\mathbb{S}} \subset\mathbb{S}\subset\mathbb{L}.
\eqno (4.2) 
$$

\section{Операция проектирования на вложенное пространство}

Из соотношений (4.1) следует, что система функционалов $\{\widetilde g^{(s)}\}_{s\in   \mathbb{Z}}$ биортогональна системе функций $\{\widetilde \omega_i\}_{i\in   \mathbb{Z}}$,
$$
\langle \widetilde g^{(s)}, \widetilde \omega_i \rangle = \delta_{s,i} \quad\forall s, i \in   \mathbb{Z}. 
\eqno (5.1) 
$$

Рассмотрим операцию $P$ проектирования пространства $\mathbb{S}$ на подпространство $\widetilde{\mathbb{S}}$, которая действует по формуле
$$
P u \defabove \sum_{s\in   \mathbb{Z}} \langle \widetilde g^{(s)}, u \rangle \omega_s,  \quad\forall u \in   \mathbb{S}(X,\varphi).
\eqno (5.2)
$$
Ясно, что если $t \in (\widetilde x_k, \widetilde x_{k+1})$ фиксировано, то правая часть формулы (5.2) имеет не более $m + 1$ слагаемого:
$$
P u(t) \defabove \sum_{s=k-m}^k \langle \widetilde g^{(s)}, u \rangle \omega_s(t) , \quad\forall t \in (\widetilde x_k, \widetilde x_{k+1}). 
\eqno (5.3)
$$

Операция проектирования $P$ определяет прямую сумму
$$
\mathbb{S} = \widetilde{\mathbb{S}}\dotplus  \mathbb{W}. 
\eqno (5.4)
$$

\section{Декомпозиция и реконструкция}

Пусть $\cc = (\ldots , c_{-2}, c_{-1}, c_0, c_1, c_2, \ldots)$ --- исходный поток числовой информации. Рассмотрим функцию
$$ 
u(t) \defabove \sum_j c_j\omega_j (t).
\eqno (6.1)
$$ 
Ее проекция $\widetilde u \defabove P u$ на пространство $\widetilde{\mathbb{S}}$ может быть представлена в форме
$$
\widetilde  u = \sum_i a_i \widetilde\omega_i. 
\eqno (6.2)
$$ 
Итак, имеем так называемый основной поток
$$ \aa \defabove (\ldots, a_{-2}, a_{-1}, a_0, a_1, a_2, \ldots),
$$
который соответствует укрупнению $\widetilde X$ сетки $X$, а также вспесковый поток $ \bb \defabove (\ldots, b_{-2}, b_{-1}, b_0, b_1, b_2, \ldots)
$, который определяется разложением разности $w \defabove u - \widetilde u$ по базису пространства $\mathbb{S}$: $w = \sum_s b_s\omega_s$ (см. формулы (5.3) -- (5.4), (6.1) -- (6.2)). 

Переход от исходного потока $\cc$ к потокам $\aa$ и $\bb$ называется {\it декомпозицией}, а обратный переход называется {\it реконструкцией}. 

Формулы декомпозиции могут быть представлены в форме
$\aa =\mathfrak{Q}\cc$, $\bb = \cc-\mathfrak{P}^T\mathfrak{Q}\cc$, 
а формулы реконструкции --- в форме 
$\cc = \bb+\mathfrak{P}^T\aa$. 
Здесь $\mathfrak{P}$ и $\mathfrak{Q}$ --- матрицы, которые называются матрицами сужения и продолжения соответственно. 

В частном случае для $m = 1$, $\varphi(t) = (1, t)^T$, $X\setminus \widetilde X = \{x_{k+1}\}$ формулы декомпозиции имеют вид
$$
a_i = c_i \quad\hbox{при}\quad i \leq k - 1, \quad 
a_i = c_{i+1} \quad\hbox{при}\quad i \geq k,
\eqno (6.3)
$$
$$ 
b_j = 0 \quad\hbox{при}\quad j \ne k, \quad  
b_k = - \frac{x_{k+2} - x_{k+1}} {x_{k+2} - x_k} \cdot c_{k-1} + c_k - \frac{x_{k+1} - x_k} {x_{k+2} - x_k} \cdot c_{k+1},
\eqno (6.4)
$$
а формулы реконструкции могут быть представлены в форме
$$ 
c_j = a_j + b_j \quad\hbox{при}\quad j \leq k - 1,
\eqno (6.5)
$$
$$
c_k = \frac{x_{k+2} - x_{k+1}} {x_{k+2} - x_k} \cdot a_{k-1} + \frac{x_{k+1} - x_k} {x_{k+2} - x_k} \cdot a_k + b_k,
\eqno (6.6)
$$
$$
c_j = a_{j-1} + b_j \quad\hbox{при}\quad j \geq k + 1.
\eqno (6.7)
$$ 

Заметим, что если отрезок $[a, b]$ содержится в интервале $(\alpha,\beta)$, то все предыдущие построения справедливы для сужения рассматриваемых функций на этот отрезок; при этом рассматриваемые сетки, а также исходный, основной и всплесковый потоки оказываются конечными. 

\section{О численной устойчивости декомпозиции и реконструкции}

{\bf Определение.} {\it Сетка $X \defabove \{x_i\}_{i \in   \mathbb{Z}}$ вида (1.1) -- (1.2) называется локально квазиравномерной, если существует число $K_0 \geq 1$ такое, что справедливо соотношение
$$
K_0^{-1}\leq \frac{x_{s+1}-x_s} {x_s-x_{s-1}}\leq K_0 \quad\forall s \in   \mathbb{Z}.
\eqno (7.1)
$$
Множество сеток, удовлетворяющих неравенствам (7.1), называется классом локально квазиравномерных сеток типа $K_0$. Этот класс сеток обозначается $\mathcal{X}(K_0)$. Множество всех сеток обозначаем $\mathcal{X}(\infty)$}.

Заметим, что $\mathcal{X}(1)$ --- множество равномерных сеток (при этом, очевидно, должно быть $\alpha = -\infty$, $\beta = +\infty$). Дальше считаем, что $K_0 \geq 1$.

{\bf Лемма 1.} {\it Если сетка (1.1) -- (1.2) является локально квазиравномерной сеткой типа $K_0$, то
$$
(1 + K_0)^{-1}\leq \frac{x_{k+1}-x_k} {x_{k+2}-x_k}\leq (1 + K_0^{-1})^{-1},
\eqno (7.2)
$$
$$
(1 + K_0)^{-1}\leq \frac{x_{k+2}-x_{k+1}} {x_{k+2}-x_k}\leq (1 + K_0^{-1})^{-1}.
\eqno (7.3)
$$}
{\bf Доказательство.} Используя соотношение (7.1), имеем
$$
K_0^{-1} + 1 \leq \frac{x_{k+2}-x_k} {x_{k+1}-x_k} = \frac{x_{k+2}-x_{k+1}} {x_{k+1}-x_k} + 1 \leq K_0 +1.
$$
Учитывая неотрицательность всех частей этого двойного неравенства, перейдем к обратным величинам. В результате получим соотношение (7.2).

Неравенство (7.3) доказывается аналогично доказательству соотношения (7.2).  $ \blacksquare$

{\bf Лемма 2}. {\it Если сетка (1.1) -- (1.2) является локально квазиравномерной сеткой типа $K_0$, то для формул декомпозиции справедливы соотношения
$$
|a_i| = |c_i| \quad\hbox{при}\quad i \leq k - 1,\quad
|a_i| = |c_{i+1}| \quad\hbox{при}\quad i \geq k,
\eqno (7.4)
$$
$$
|b_k| \leq (1 + K^{-1}_0)^{-1}\cdot(|c_{k-1}| + |c_{k+1}|) + |c_k|. 
\eqno (7.5)
$$}
{\bf Доказательство.} Соотношения (7.4) вытекают из формул (6.3), а соотношения (7.5) получаются из формулы (6.4) и неравенств (7.2) -- (7.3).  $\blacksquare$

{\bf Лемма 3.} {\it Если сетка (1.1) -- (1.2) является локально квазиравномерной сеткой типа $K_0$, то для формул реконструкции верны неравенства 
$$
|c_j| \leq |a_j| + |b_j| \quad\hbox{при}\quad j \leq k - 1,
\eqno (7.6)
$$
$$
|c_k| \leq (1 + K^{-1}_0)^{-1} \cdot (|a_{k-1}| + |a_k|) + |b_k| 
\eqno (7.7)
$$
$$
|c_j| \leq |a_{j-1}| + |b_j| \quad\hbox{при}\quad j \geq k + 1. 
\eqno (7.8)
$$}
{\bf Доказательство.} Соотношения (7.6) и (7.8) вытекают из формул (6.5) и (6.7), а соотношение (7.7) следует из формулы (6.6) и неравенств (7.2) -- (7.3). $\blacksquare$

Введем обозначения
$$
\|\aa\| \defabove \sup_{i\in   \mathbb{Z}} |a_i|,\quad
\|\bb\| \defabove \sup_{i\in   \mathbb{Z}} |b_i|, \quad 
\|\cc\| \defabove \sup_{i\in   \mathbb{Z}} |c_i|.
$$

{\bf Теорема 1.} (Об устойчивости декомпозиции) {\it Если $X$ --- локально квазиравномерная сетка типа $K_0$, то справедливы неравенства
$$
\|\aa\| \leq \|\cc\|,
\eqno (7.9)
$$
$$
\|\bb\| \leq (2(1 + K^{-1}_0)^{-1} + 1)\|\cc\|.
\eqno (7.10)
$$}
{\bf Доказательство.} Из (7.4) имеем
$$
|a_i| = |c_i| \leq \|\cc\| \quad\hbox{при}\quad i \leq k - 1,\quad 
|a_i| = |c_{i+1}| \leq \|\cc\|  \quad\hbox{при}\quad i \geq k,
$$
откуда вытекает неравенство (7.9). Из (7.5) выводим
$$
|b_k| \leq 2(1 + K^{-1}_0)^{-1} \cdot \|\cc\| + \|\cc\|.
$$
Таким образом, неравенство (7.10) доказано. $\blacksquare$

{\bf Теорема 2.} (Об устойчивости реконструкции) {\it Если $X$ --- локально квазиравномерная сетка типа $K_0$, то справедливы неравенства
$$
\|\cc\| \leq \max\{2(1 + K^{-1}_0)^{-1}, 1\} \cdot \|\aa\| + \|\bb\|.
\eqno (7.11)
$$}
{\bf Доказательство.} Из неравенств (7.6) и (7.6) имеем
$$
|c_j| \leq \|\aa\| + \|\bb\| \quad\hbox{при}\quad j \in   \mathbb{Z}\setminus \{k\},
$$
а из соотношения (7.7) имеем
$$
|c_k| \leq 2(1 + K^{-1}_0)^{-1} \cdot\|\aa\| + \|\bb\| .
$$ 
Из последних двух неравенств следует соотношение (7.11). $\blacksquare$ 


\section{Заключение}
В работе установлены критерии устойчивости сплайн-всплесковых разложений. Доказаны теоремы об устойчивости декомпозиции и реконструкции.


\begin{thebibliography}{8}

\bibitem{dem2003} Демьянович Ю. К. Всплески \& минимальные сплайны. СПб., 2003.

\bibitem{dem2013} Демьянович Ю. К. Теория сплайн-всплесков. СПб., 2013.  

\end{thebibliography}

\end{document}
